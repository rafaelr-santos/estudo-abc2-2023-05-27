\section{Capítulo 2}

\begin{frame}
	\frametitle{Tópicos chaves do Capítulo}

	\begin{itemize}
		\item Retomada da crítica do modelo de discipulado "atual"
		\item Caracterização de práticas em tênues e densas
		\item Ideias de Formação, Deformação e Contraformação.
	\end{itemize}
\end{frame}

\begin{frame}
	\frametitle{Crítica}

	\begin{itemize}
		\item <1> Estamos em meio a diversas intituições litúrgicas. E nenhuma delas é neutra.
		\item <2> Portanto, nosso desejo está o tempo todo sendo formado, inclusive por intituções litúrgicas seculares. Ex: Marketing de empresas de consumo, no geral (Alimento, entreterimento, etc...)
		\item <3> A igreja responde a essa formação do desejo com ideia.
	\end{itemize}
\end{frame}

\begin{frame}
	\frametitle{Exemplo fora da igreja}
    \begin{figure}[h]
        \includegraphics[width=8cm]{imagens/cigarro.png}
    \end{figure}
\end{frame}

\begin{frame}
	\frametitle{Exemplo fora da igreja}
    \begin{figure}[h]
        \includegraphics[width=8cm]{imagens/transito.png}
    \end{figure}
\end{frame}

\begin{frame}
	\frametitle{Exemplos de dentro da igreja}
    \begin{itemize}
        \item Falta de engajamento na missão.
        \item Rotina devocional de Oração e Leitura.
        \item Adultério, Violência, etc...
    \end{itemize}
\end{frame}

\begin{frame}
	\frametitle{Respostas da Igreja}
    \begin{itemize}
        \item Fóruns
        \item Palestras
        \item Congressos 
    \end{itemize}
\end{frame}

\begin{frame}
	\frametitle{Proposta}
    \begin{itemize}
        \item <1>E um outro problema: Podemos até amar as coisas certas, mas frequentemente as colocamos na ordem/prioridade errada.\footnotemark
        \item <2>"Se você quer construir um navio, não chame as pessoas para juntar madeira, mas ensine a desejar o oceano." \footnotemark
        \item <2>Direto ao ponto: Reordenar e formular desejos. algumas coisa temos que abandonar, outras priorizar em outra ordem.
    \end{itemize}

    \footnotetext[6]{Ideia de Agostinho}
    \footnotetext[7]{Citação do Pequeno Príncipe pelo J.K.A Smith no Você é Aquilo eu Ama}
\end{frame}


\begin{frame}
	\frametitle{Caracterização de Práticas}

    \begin{itemize}
        \item <1> Portanto, se temos que reordenar nossos desejos e somos aquilo que amamos, então caraterizar o impacto de algumas práticas na nossa formação pare uma boa ideia.
        \item <2> Se soubermos como as coisas que fazemos impactam na nossa formação, fica mais fácil de saber o que abandonar ou o que repriorizar.
    \end{itemize}
\end{frame}

\begin{frame}
	\frametitle{Proposta de Categorização}

    \begin{itemize}
        \item Práticas tênues/leves/rasas/corriqueiras : Práticas que são meios para outros fins e que tem impacto moderado sobre nossa identidade. Ex. : Escovar os dentes.
        \item Práticas densas: Pŕaticas que tem impacto grande sobre nossa identidade. Ex. frequentar uma igreja.
    \end{itemize}
\end{frame}

\begin{frame}
	\frametitle{É uma prática densa ou tênue?}

    \begin{itemize}
        \item Contexto: Uma pessoa comum, que mora sozinha e trabalha. Também não problemas exóticos. E está satifeita com a vida.
        \item Prática: Lavar louça. Não consme nenhum tipo de conteúdo audiovisual enquanto lava.
    \end{itemize}
\end{frame}


\begin{frame}
	\frametitle{É uma prática densa ou tênue?}

    \begin{itemize}
        \item Contexto: Uma pessoa que não problemas financeiros nem psciológicos; e está satifeita com a vida.
        \item Prática: Faz questão de cortar o cabelo somente no segundo dia da fase crescente da lua; e em um cabelereiro caro.
    \end{itemize}
\end{frame}

\begin{frame}
	\frametitle{É uma prática densa ou tênue?}

    \begin{itemize}
        \item Contexto: Uma pessoa que mora sozinha, não conversa muito com os pais e tem com problemas financeiros sérios. Mas que consegue dar seus pulos quando falta dinheiro. Seja de modo lícito ou não.
        \item Prática: Assistir um filme em um shopping, na poltrona VIP e comendo pipoca. É uma prática muito frequente em sua vida.
    \end{itemize}
\end{frame}

\begin{frame}
	\frametitle{É uma prática densa ou tênue?}

    \begin{itemize}
        \item Contexto: Um homem empregado, casado e com dois filhos pequenos, mora somente com esposa e os filhos. Tem condição financeira boa.
        \item Prática: Assim que chega do trabalho, já liga o videgame para desestressar, logo depois de cumprimentar rapidamente a esposa e os filhos.
    \end{itemize}
\end{frame}

\begin{frame}
	\frametitle{É uma prática densa ou tênue?}

    \begin{itemize}
        \item Contexto: Uma criança com pais casados, tem 10 anos. Tem condição financeira boa. Não aparenta ter dificuldades na escola, e também consegue interagir com outras crianças naturalmente.
        \item Prática: Assiste documentários sobre serial killers frequentemente.(Ele tem acesso às plataformas de streaming dos pais)
    \end{itemize}
\end{frame}

\begin{frame}
	\frametitle{Caracterização de práticas}

    \begin{itemize}
        \item <1>Na verdade, é complicado de caracterizar as práticas.
        \item <1>Em determinando momento o autor diz: "Pode pender mais para um lado do que para o outro". Então é um gradiente.
        \item <2> Exemplos do autor: Assistir um jornal a noite (tênue), e ouvir o rádio na rodovia (denso)
        \item <3> Seguindo a lógica que as instituções liturgicas não são neutras. Não existem práticas puramente tênues.
    \end{itemize}
\end{frame}

\begin{frame}
	\frametitle{Caracterização de práticas}

    \begin{itemize}
        \item A ideia é colocar em prática o que Jesus disse: "Conheceremos as árvores pelos frutos"
    \end{itemize}
\end{frame}