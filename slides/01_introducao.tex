\section{Introdução}

\begin{frame}
	\frametitle{Contexto do Livro}

	\begin{itemize}
		\item 1° dos Livros da série Liturgias Culturais. Os outros são: Imaginando o reino: a dinâmica do culto e Aguardando o Rei: reformando a teologia.
		\item 1° Edição de 2018.
	\end{itemize}
\end{frame}

\begin{frame}
	\frametitle{Objetivo}

	\begin{itemize}
		\item <1> ..."transmitir a estudantes uma concepção de como deveria ser o ensino autêntico e integralmente cristão"...(pg. 9, Prefácio)
		\item <2> ..."Seria uma honra se"... (pg. 9) servisse para acrescentar na discução da forma da adoração de comunidades locais.
		\item <3> ..."convite à re-visão da educação como projeto de formação, e não apenas de informação"...(pg. 18)
	\end{itemize}
\end{frame}

\begin{frame}
	\frametitle{Público}

	\begin{itemize}
		\item <1> Principal:  Estudantes e Professores.
		\item <1> Secundários: Pastores, Ministros que atuam nos \emph{campi} universitários, líderes de adoração, responsáveis pela forma da adoração cristã nas igrejas em geral.
		\item <2> Em particular, o livro cita o contexto universitário mais que os outros contexto.
	\end{itemize}
\end{frame}

\begin{frame}
	\frametitle{Lembrando de algumas definições}

	\begin{itemize}
		\item <1>Liturgia: Práticas Formativas. Pedagogias Formativas do Desejo. Rituais que moldam e formam o coração\footnotemark.
		\item <2>Instituição Litúrgica: Conglomerado de práticas. Contexto em que existem liturgias. Ex. :Shopping, família, etc...
		\item <3>Adoração, desejo e amor são palavras que se confundem um pouco.
		\item <4>Ritual, prática, liturgia e pedagogia são palavras que se confundem um pouco.
		\item <5>As liturgias estão presentes em todos os lugares, e não existem liturgias neutras.
	\end{itemize}
	\footnotetext[1]{Este último tirei do Você é aquilo que Ama}
\end{frame}

\begin{frame}
	\frametitle{Tentando explicar as palavras}
    \begin{figure}[h]
        \includegraphics[width=10cm]{imagens/palavras.png}
    \end{figure}
\end{frame}

\begin{frame}
	\frametitle{Ideia chave do texto}

	\begin{itemize}
		\item <1>A forma com que o discipulado é entendido e praticado está equivocada: Muita informação, pouca prática\footnotemark./
		\item <2>Exemplos: Culto com slides; Formato das aulas nas universidades.
	\end{itemize}

	\footnotetext[2]{Coisas pensantes}
\end{frame}

\begin{frame}
	\frametitle{Outra ideia chave do texto}

	\begin{itemize}
		\item <1>Você não é aquilo que pensa, pois nem tudo que pensa coloca em prática. Não somos somente o mal que praticamos\footnotemark.
		\item <2>Uma outra proposta: "Você é aquilo que ama, pois está voltado para aquilo que deseja"\footnotemark.
	\end{itemize}

	\footnotetext[3]{Um detalhe: isso esta de acordo com a ideia anterior}
	\footnotetext[4]{Este tirei do Você é aquilo que Ama}
\end{frame}

\begin{frame}
	\frametitle{Implicação da ideia anterior}

	\begin{itemize}
		\item <1>Uma questão importante dos discipulado é: Conhecemos para adorar, ou adoramos antes de conhecer?\footnotemark
		\item <2>Amamos antes de conheçer
	\end{itemize}

	\footnotetext[5]{Conhecer-te ou invocar-te? Quem procura encontra-o, quem encontra louvá-lo-á}
\end{frame}

