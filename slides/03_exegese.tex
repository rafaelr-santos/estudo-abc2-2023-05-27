\section{Capítulo 3}

\begin{frame}
	\frametitle{Tópicos chaves do Capítulo e Motivação}

	\begin{itemize}
		\item <1>Exegese cultural de algumas situações:
        \begin{enumerate}
            \item Shopping
            \item "Entreterimento Militar"
            \item Universidade
        \end{enumerate}
        \item <2>Bom, agora que sabemos caraterizar as práticas, vamos analisar tudo em contexto, e perceber o papel das instituções liturgicas na formação.
	\end{itemize}
\end{frame}

\begin{frame}
	\frametitle{Ideia da Exegese Cultural}

	\begin{itemize}
        \item Hermenêutica: Regras e boas práticas de interpretação e análise do texto.
        \item Exegese: Análise e Interpretação do texto.
	\end{itemize}
\end{frame}

\begin{frame}
	\frametitle{Mudança de paradigma}

	\begin{itemize}
        \item <1>Qual é a ideia/conceito que aquela determinada intitução passa??
        \item <2>Que tipo de pessoa as pessoa se tornam depois de ter contato com determinada instituição liturgica?
        \item <2>Quais são as práticas densas empregadas em determinado contexto qual a consequencia dessas práticas naqueles que a praticam?
        \item <3>Quais são as intituições liturgicas envolvidas no contexto?
	\end{itemize}
\end{frame}